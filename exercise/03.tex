\exercise{3}

Der Messfehler eines Ger\"ats zur Entfernungsmessung wird als normalverteilt
mit Mittel $0$ und Standardabweichung $\sigma$ angenommen.

\begin{enumerate}[label=\alph*)]
 \item Wie gro{\ss} darf $\sigma$ maximal sein,
       damit der Messwert mit $99.9\%$ Sicherheit
       um h\"ochstens $1$ mm vom wahren Wert abweicht?
 \item Wann ist die Annahme eines normalverteilten Fehlers mit Mittel $0$
       gerechtfertigt?
\end{enumerate}


\answer{a)}

\answer{b)}
